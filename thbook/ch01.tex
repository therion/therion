\chapter Introduction.

Therion is a tool for cave surveying. Its purpose is to help 

\list
* archive survey data on computer in a form as close to the original notes and 
  sketches as possible and retrieve them in a flexible and efficient way;
* draw a nice up-to-date plan or elevation map; 
* create a realistic 3D model of the cave.
\endlist

It runs on Unix, Linux, MacOS X and Win32 operating systems. 
Source code and precompiled binaries for Linux and Win32 
systems are available on the Therion web page (\www{|http://therion.speleo.sk|}%
{http://therion.speleo.sk}).

Therion is distributed under the \www{GNU General Public License}%
{http://www.gnu.org/}.

\subchapter Why Therion?.

In the 1990s we've done a lot of caving and cave surveying. Some computer 
programs existed which displayed survey shots and stations after loop closure and 
error elimination. These were a great help, especially for large and complicated 
cave systems. We used the output of one of them---TJIKPR---as a background 
layer with survey stations for hand\discretionary{-}{-}{-}drawn maps. After finishing a huge 166 
page Atlas of Dead Bats Cave, we soon had a problem: we found new passages
connecting between known passages and surveyed them. After processing in TJIKPR, 
the new loops influenced the position of the old surveys;
most survey stations now had a slightly different position from before due to the
changed error distribution. So we could either draw the whole Atlas again, 
or accept that the location of some places was not accurate---in the case of loops 
with a length of approximately 1\,km there were sometimes errors of about 
10\,m---and try to distort the new passages to fit to old ones.

These problems remained when we tried to draw maps using some CAD programs. It 
was always hard to add new surveys without adapting the old ones to the newly 
calculated positions of survey stations in the whole cave. We found no program that was 
able to draw an up-to-date complex map (i.e.~not just survey shots with LRUD envelope), 
in which the old parts are modified according to the most recent known coordinates 
of survey stations.

In 1999 we begun to think about creating own program for map drawing. We knew 
about programs which were perfectly suited for particular sub-tasks. There 
was \MP, a high level programming language for vector graphics description, 
Survex for excellent processing of survey shots, and \TeX\ for typesetting the 
results. We had only to glue them together. By Xmas 
1999 we had a minimalistic version of Therion working for the first time. This 
consisted only of about 32\,kB of Perl scripts and \MP\ macros but served the 
purpose of showing that our ideas were implementable.

During 2000--2001 we searched for the optimal format of the input data, programming 
language, concept of interactive map editor and internal algorithms with the 
help of Martin Sluka (Prague) and Martin Heller (Z\"urich). In 2002 we were able to 
introduce the first really usable version of Therion, which met our requirements.


\subchapter Features.

Therion is a command-line application. It processess input files, which 
are---including 2D maps---in text format, and creates files with 2D maps or 
3D model as the output.

The syntax of input files is described in detail in later chapters. 
You may create these files in an arbitrary plain text editor like 
{\it ed} or {\it vi}. They contain instructions for Therion like 

|point 1303 1004 pillar| 

where {\tt point} is a keyword for point symbol 
followed by its coordinates and a symbol type specification.

Hand-editing of such files is not easy---especially when you draw maps, you 
need to think in spatial (Cartesian coordinate) terms. Thus there is a special 
GUI for Therion called XTherion. XTherion works as an advanced text editor, map 
editor (where maps are drawn fully interactively), and compiler (which runs
Therion on the data).

It may look quite complicated, but this approach has a lot of advantages:

\list
* There is strict separation of data and visualization. The data files specify 
  only what is where, not what it looks like. The visual representation 
  is added by \MP\ in later phases of data processing. (It's very 
  similar concept to XML data representation.) 
  
  This makes it possible to change map symbols used without changing the 
  input data, or merge more maps created by different persons in different 
  styles into one map with unified map symbols set.

  2D maps are adapted for particular output scale (level of abstraction,
  non-linear scaling of symbols and texts)
  
* All data are relative to survey station positions. If the coordinates
  of survey stations are changed in the process of loop closure, then all relevant
  data is moved correspondingly, so the map is always up-to-date.

* Therion is not dependent on particular operating system, character encoding
  or input files editor; input files will remain human readable

* It's possible to add new output formats

* 3D model will be generated from 2D plan, elevation and 
  cross-sections data to get a realistic 3D model without entering
  too much data

* although the support for WYSIWYG is limited, you get what you want (WYWIWYG)
\endlist

\subchapter Software requirements.

``A program should do one thing, and do it well.'' (Ken Thompson) 
Therefore we use some valuable external programs, which are related to the problems of cave 
surveying, typesetting and data visualization. Therion can then do its task 
much better than if it was a standalone application in which you could calibrate 
your printer or scanner and with one click send e-mail with your data.

Therion needs:
\list
* recent Survex (necessary if you want to process any survey data)
  \path|http://www.survex.com|
  
* \TeX\ distribution with at least \TeX\ with Plain format, 
  recent pdf-$\varepsilon$-\TeX, and \MP\ with accompanying programs. 
  (\TeX\ and friends is necessary only if you want to create 2D maps.)
  Plain should be the default format for \TeX; 
  Therion doesn't use La\TeX\ or other available formats. If you 
  have any doubts, install any full distribution of \TeX. (TeXlive for any 
  platform, teTeX for Unix, fpTeX or MikTeX for Windows.) If you're new
  to \TeX, visit the homepage of {\it \TeX\ Users Group} at 
  \path|http://www.tug.org|

* Tcl/Tk 8.4.3 and newer (|http://www.tcl.tk|) with BWidget widget set 
  (\path|http://sourceforge.net/projects/tcllib|). Tcl/Tk is only required for 
  XTherion. For Windows try ActiveTcl from |http://www.activestate.com|.
\endlist

Get all this working before installing Therion. We don't provide any support 
concerning this. All these programs are well documented and accessible on the 
net.


\subchapter Installation.

Primary Therion distribution is the source code. This needs be compiled 
according to instructions in the {\it Appendix}.

For users' convenience there are also precompiled binaries for Linux and 
Windows. 

{\bf Installation on Linux (*.tar.gz package):}

Unpack the archive and run the installation script `install.'

{\bf Installation on Windows:}

Run the setup program and follow instructions.


\subsubchapter Setting-up environment.

Therion reads settigs from the initialization file. Default settings should 
work fine for users using only ASCII (non-accented latin) characters, 
standard \TeX\ and \MP. 

If you want to use accented latin or non-latin characters, edit initialization 
file. Instructions on how to do this are in the {\it Appendix}.


\subchapter How does it work?.

So, now it's clear what Therion needs, let's have a look at the way 
it interacts with all these programs:

\pic{mp/schema.1}

DON'T PANIC! When your system is set-up right the majority of this is hidden from 
the user and all necessary programs are run automatically by Therion. 

For working with Therion it is enough to know that you have to create input data 
(best done with XTherion), run Therion, and display output files 
(3D model, PDF map, log file) in the appropriate program. 

For those who want to understand more about it, here is a brief explanation of 
the above flowchart. Program names are in roman font, data files in italics. 
Arrows show data flow between programs. Temporary data files are not shown. 
Meaning of colors:

\list
* black---Therion programs and macros (XTherion is written in Tcl/Tk,
  so it needs this interpreter to run)
* blue---Survex program
* red---\TeX\ package
* green---input files created by the user and output files created by Therion
\endlist

Therion itself does the main task. It reads the input files, interpretes them, 
exports survey data to Survex, which does the loop closure. Then it reads back 
coordinates of survey stations and transforms all other data (e.g.~2D maps) 
according to them. Therion exports data for 2D maps in \MP\ format. \MP\ gives 
the actual shape to abstract map symbols according to map symbol definitions; it
creates a lot of PostScript files with small fragments of the cave. These are 
read back and converted to a PDF-like format, which forms input data 
for pdf\TeX. Pdf\TeX\ does all the typesetting and creates a PDF file of the cave 
map. 

Therion may also export simple 3D model (survey shots only) in Survex *.3d 
and Compass *.plt formats. [Complex 3D in the future]

Centreline may be exported for further processing in any SQL 
database.

For displaying the results you need:
\list
* Aven or xcaverot from Survex package for viewing the simple 3D model
* any PDF viewer like Acrobat Reader, ghostscript or xpdf for displaying  
  2D maps
* SQL database client to process exported database
\endlist



\subchapter First run.

After explaining the basic principles of Therion it's a good idea to try it
on the example data.

\list
* Download the sample data from Therion web page and unpack it somewhere on
  your computer's hard drive.
* Run XTherion (under Unix and MacOS\,X by typing `xtherion' in the command 
  line, under Windows there is a shortcut in the {\it Start} menu).
* Open the file `thconfig' from the sample data directory in the `Compiler'
  window of XTherion
* Press `F9' or `compile' in the menu to run Therion on the data---you'll get
  some messages from Therion, Survex, \MP\ and \TeX.  
* PDF maps and 3D model are created in the data directory.  
\endlist

Additionally, you may open survey data files (*.th) in the `Text editor' window 
and map data files (*.th2) in the `Map editor' window of XTherion. Although the 
data format may look confusing for the first time, it will be explained in the 
following chapters.

\endinput
